\documentclass[a4paper, twoside]{report}

\begin{document}
\chapter*{Building Gene Regulatory Network}
\section*{Introduction}

Developmental gene regulatory networks (GRN) plays a crucial role in the development of every single individual, especially multicellular. 
They regulate developmental processes not only during embryogenesis, but also affect post-embrionic morphogenetic processes like regeneration, asexual reproduction and growth.

GRN can be viewed as highly complicated network of protein-gene interacctions. 
As a proteins in most of the cases act transcription factors (TFs) or protein kinases. 
TFs are the proteins which has special DNA-binding motifs in their structure. 
This allows them to bind to regulatory elements in adjacent to gene regions (cis-regulatory modules) and regulate gene expression.
Depending on type of the interaction (activation/repression) there are can be several outcomes.
If TF binds to enchancer region, gene expression increse.
Opposite to that, if protein binds to silencer region then gene decrese its expression.
Also there can be more complicated types of interaction. For example protein could be insulator, enhancer-blocker or a barrier.
As a result most of processes during development can be seen as complex gene-protein interaction network with nodes and edges. 
Nodes shows genes or TFs, edges illustrate type of interaction.
Such graphs in some cases reach vary big complexity.
But here still remains a questions: how we can reconstruct such networks? 
What kind of data should we use? 
And what are the potencial outcomes of reasearch such gene interactions? 

\section*{Gene regulatory networks}

\section*{Experimental approach}

\subparagraph{Temporal expression}

\subparagraph{Spatial expression}

\subparagraph{Linkages}

\subparagraph{Validation}

\subparagraph{Visualisation}

\section*{NGS approach}

\section*{Conslusion}


Therefore research of such a complex regulatory systems can shed more light on developmenatal processes and help in the curing of molecular genetic diseases associated with the violation of the networks.

\end{document}



